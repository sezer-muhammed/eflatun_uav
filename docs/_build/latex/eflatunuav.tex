%% Generated by Sphinx.
\def\sphinxdocclass{report}
\documentclass[letterpaper,10pt,english]{sphinxmanual}
\ifdefined\pdfpxdimen
   \let\sphinxpxdimen\pdfpxdimen\else\newdimen\sphinxpxdimen
\fi \sphinxpxdimen=.75bp\relax
\ifdefined\pdfimageresolution
    \pdfimageresolution= \numexpr \dimexpr1in\relax/\sphinxpxdimen\relax
\fi
%% let collapsible pdf bookmarks panel have high depth per default
\PassOptionsToPackage{bookmarksdepth=5}{hyperref}

\PassOptionsToPackage{booktabs}{sphinx}
\PassOptionsToPackage{colorrows}{sphinx}

\PassOptionsToPackage{warn}{textcomp}
\usepackage[utf8]{inputenc}
\ifdefined\DeclareUnicodeCharacter
% support both utf8 and utf8x syntaxes
  \ifdefined\DeclareUnicodeCharacterAsOptional
    \def\sphinxDUC#1{\DeclareUnicodeCharacter{"#1}}
  \else
    \let\sphinxDUC\DeclareUnicodeCharacter
  \fi
  \sphinxDUC{00A0}{\nobreakspace}
  \sphinxDUC{2500}{\sphinxunichar{2500}}
  \sphinxDUC{2502}{\sphinxunichar{2502}}
  \sphinxDUC{2514}{\sphinxunichar{2514}}
  \sphinxDUC{251C}{\sphinxunichar{251C}}
  \sphinxDUC{2572}{\textbackslash}
\fi
\usepackage{cmap}
\usepackage[T1]{fontenc}
\usepackage{amsmath,amssymb,amstext}
\usepackage{babel}



\usepackage{tgtermes}
\usepackage{tgheros}
\renewcommand{\ttdefault}{txtt}



\usepackage[Bjarne]{fncychap}
\usepackage{sphinx}

\fvset{fontsize=auto}
\usepackage{geometry}


% Include hyperref last.
\usepackage{hyperref}
% Fix anchor placement for figures with captions.
\usepackage{hypcap}% it must be loaded after hyperref.
% Set up styles of URL: it should be placed after hyperref.
\urlstyle{same}

\addto\captionsenglish{\renewcommand{\contentsname}{Contents:}}

\usepackage{sphinxmessages}
\setcounter{tocdepth}{1}



\title{eflatun uav}
\date{May 16, 2023}
\release{0.0.2}
\author{Muhammed Sezer}
\newcommand{\sphinxlogo}{\vbox{}}
\renewcommand{\releasename}{Release}
\makeindex
\begin{document}

\ifdefined\shorthandoff
  \ifnum\catcode`\=\string=\active\shorthandoff{=}\fi
  \ifnum\catcode`\"=\active\shorthandoff{"}\fi
\fi

\pagestyle{empty}
\sphinxmaketitle
\pagestyle{plain}
\sphinxtableofcontents
\pagestyle{normal}
\phantomsection\label{\detokenize{index::doc}}


\sphinxstepscope


\begin{savenotes}\sphinxattablestart
\sphinxthistablewithglobalstyle
\sphinxthistablewithnovlinesstyle
\centering
\begin{tabulary}{\linewidth}[t]{\X{1}{2}\X{1}{2}}
\sphinxtoprule
\sphinxtableatstartofbodyhook
\sphinxAtStartPar
{\hyperref[\detokenize{generated/eflatun_uav:module-eflatun_uav}]{\sphinxcrossref{\sphinxcode{\sphinxupquote{eflatun\_uav}}}}}
&
\sphinxAtStartPar

\\
\sphinxbottomrule
\end{tabulary}
\sphinxtableafterendhook\par
\sphinxattableend\end{savenotes}

\sphinxstepscope


\chapter{eflatun\_uav}
\label{\detokenize{generated/eflatun_uav:module-eflatun_uav}}\label{\detokenize{generated/eflatun_uav:eflatun-uav}}\label{\detokenize{generated/eflatun_uav::doc}}\label{\detokenize{generated/eflatun_uav::doc}}\index{module@\spxentry{module}!eflatun\_uav@\spxentry{eflatun\_uav}}\index{eflatun\_uav@\spxentry{eflatun\_uav}!module@\spxentry{module}}\subsubsection*{Modules}


\begin{savenotes}\sphinxattablestart
\sphinxthistablewithglobalstyle
\sphinxthistablewithnovlinesstyle
\centering
\begin{tabulary}{\linewidth}[t]{\X{1}{2}\X{1}{2}}
\sphinxtoprule
\sphinxtableatstartofbodyhook
\sphinxAtStartPar
{\hyperref[\detokenize{generated/eflatun_uav.filters:module-eflatun_uav.filters}]{\sphinxcrossref{\sphinxcode{\sphinxupquote{eflatun\_uav.filters}}}}}
&
\sphinxAtStartPar
Filter implementations for moving objects
\\
\sphinxhline
\sphinxAtStartPar
{\hyperref[\detokenize{generated/eflatun_uav.helpers:module-eflatun_uav.helpers}]{\sphinxcrossref{\sphinxcode{\sphinxupquote{eflatun\_uav.helpers}}}}}
&
\sphinxAtStartPar

\\
\sphinxbottomrule
\end{tabulary}
\sphinxtableafterendhook\par
\sphinxattableend\end{savenotes}

\sphinxstepscope


\section{eflatun\_uav.filters}
\label{\detokenize{generated/eflatun_uav.filters:module-eflatun_uav.filters}}\label{\detokenize{generated/eflatun_uav.filters:eflatun-uav-filters}}\label{\detokenize{generated/eflatun_uav.filters::doc}}\index{module@\spxentry{module}!eflatun\_uav.filters@\spxentry{eflatun\_uav.filters}}\index{eflatun\_uav.filters@\spxentry{eflatun\_uav.filters}!module@\spxentry{module}}
\sphinxAtStartPar
Filter implementations for moving objects
\subsubsection*{Classes}


\begin{savenotes}\sphinxattablestart
\sphinxthistablewithglobalstyle
\sphinxthistablewithnovlinesstyle
\centering
\begin{tabulary}{\linewidth}[t]{\X{1}{2}\X{1}{2}}
\sphinxtoprule
\sphinxtableatstartofbodyhook
\sphinxAtStartPar
{\hyperref[\detokenize{generated/eflatun_uav.filters:eflatun_uav.filters.BaseFilter}]{\sphinxcrossref{\sphinxcode{\sphinxupquote{BaseFilter}}}}}(input\_size, output\_size)
&
\sphinxAtStartPar
A base class representing a generic filter for moving objects.
\\
\sphinxbottomrule
\end{tabulary}
\sphinxtableafterendhook\par
\sphinxattableend\end{savenotes}
\index{BaseFilter (class in eflatun\_uav.filters)@\spxentry{BaseFilter}\spxextra{class in eflatun\_uav.filters}}

\begin{fulllineitems}
\phantomsection\label{\detokenize{generated/eflatun_uav.filters:eflatun_uav.filters.BaseFilter}}
\pysigstartsignatures
\pysiglinewithargsret{\sphinxbfcode{\sphinxupquote{class\DUrole{w,w}{  }}}\sphinxcode{\sphinxupquote{eflatun\_uav.filters.}}\sphinxbfcode{\sphinxupquote{BaseFilter}}}{\sphinxparam{\DUrole{n,n}{input\_size}\DUrole{p,p}{:}\DUrole{w,w}{  }\DUrole{n,n}{List}}, \sphinxparam{\DUrole{n,n}{output\_size}\DUrole{p,p}{:}\DUrole{w,w}{  }\DUrole{n,n}{List}}}{}
\pysigstopsignatures
\sphinxAtStartPar
Bases: \sphinxcode{\sphinxupquote{object}}

\sphinxAtStartPar
A base class representing a generic filter for moving objects.

\sphinxAtStartPar
This class serves as a foundation for more specific filter implementations. It is designed to be
subclassed, and does not provide a full implementation that can be used on its own.
\index{\_\_init\_\_() (eflatun\_uav.filters.BaseFilter method)@\spxentry{\_\_init\_\_()}\spxextra{eflatun\_uav.filters.BaseFilter method}}

\begin{fulllineitems}
\phantomsection\label{\detokenize{generated/eflatun_uav.filters:eflatun_uav.filters.BaseFilter.__init__}}
\pysigstartsignatures
\pysiglinewithargsret{\sphinxbfcode{\sphinxupquote{\_\_init\_\_}}}{\sphinxparam{\DUrole{n,n}{input\_size}\DUrole{p,p}{:}\DUrole{w,w}{  }\DUrole{n,n}{List}}, \sphinxparam{\DUrole{n,n}{output\_size}\DUrole{p,p}{:}\DUrole{w,w}{  }\DUrole{n,n}{List}}}{{ $\rightarrow$ None}}
\pysigstopsignatures
\sphinxAtStartPar
Initializes the base filter with the given input and output size.
\begin{quote}\begin{description}
\sphinxlineitem{Parameters}\begin{itemize}
\item {} 
\sphinxAtStartPar
\sphinxstyleliteralstrong{\sphinxupquote{input\_size}} (\sphinxstyleliteralemphasis{\sphinxupquote{List}}) \textendash{} The size of the input state. This is usually a list where each element represents
the size of a different aspect of the input state.

\item {} 
\sphinxAtStartPar
\sphinxstyleliteralstrong{\sphinxupquote{output\_size}} (\sphinxstyleliteralemphasis{\sphinxupquote{List}}) \textendash{} The size of the output state. Similar to the input size, this is a list where each
element represents the size of a different aspect of the output state.

\end{itemize}

\end{description}\end{quote}

\end{fulllineitems}

\index{predict() (eflatun\_uav.filters.BaseFilter method)@\spxentry{predict()}\spxextra{eflatun\_uav.filters.BaseFilter method}}

\begin{fulllineitems}
\phantomsection\label{\detokenize{generated/eflatun_uav.filters:eflatun_uav.filters.BaseFilter.predict}}
\pysigstartsignatures
\pysiglinewithargsret{\sphinxbfcode{\sphinxupquote{predict}}}{}{{ $\rightarrow$ ndarray}}
\pysigstopsignatures
\sphinxAtStartPar
Predicts the next state based on the current state of the filter.

\sphinxAtStartPar
This method is intended to be overridden by subclasses.
\begin{quote}\begin{description}
\sphinxlineitem{Raises}
\sphinxAtStartPar
\sphinxstyleliteralstrong{\sphinxupquote{NotImplementedError}} \textendash{} This method must be implemented in a subclass.

\sphinxlineitem{Returns}
\sphinxAtStartPar
\begin{description}
\sphinxlineitem{The predicted next state. The size and structure of this output should match the output\_size}
\sphinxAtStartPar
specified when the filter was initialized.

\end{description}


\sphinxlineitem{Return type}
\sphinxAtStartPar
np.ndarray

\end{description}\end{quote}

\end{fulllineitems}

\index{update() (eflatun\_uav.filters.BaseFilter method)@\spxentry{update()}\spxextra{eflatun\_uav.filters.BaseFilter method}}

\begin{fulllineitems}
\phantomsection\label{\detokenize{generated/eflatun_uav.filters:eflatun_uav.filters.BaseFilter.update}}
\pysigstartsignatures
\pysiglinewithargsret{\sphinxbfcode{\sphinxupquote{update}}}{\sphinxparam{\DUrole{n,n}{input\_state}\DUrole{p,p}{:}\DUrole{w,w}{  }\DUrole{n,n}{ndarray}}}{}
\pysigstopsignatures
\sphinxAtStartPar
Updates the state of the filter based on the given input state.

\sphinxAtStartPar
This method is intended to be overridden by subclasses.
\begin{quote}\begin{description}
\sphinxlineitem{Parameters}
\sphinxAtStartPar
\sphinxstyleliteralstrong{\sphinxupquote{input\_state}} (\sphinxstyleliteralemphasis{\sphinxupquote{np.ndarray}}) \textendash{} The input state used to update the filter. The size and structure of this
input should match the input\_size specified when the filter was initialized.

\sphinxlineitem{Raises}
\sphinxAtStartPar
\sphinxstyleliteralstrong{\sphinxupquote{NotImplementedError}} \textendash{} This method must be implemented in a subclass.

\end{description}\end{quote}

\end{fulllineitems}


\end{fulllineitems}


\sphinxstepscope


\section{eflatun\_uav.helpers}
\label{\detokenize{generated/eflatun_uav.helpers:module-eflatun_uav.helpers}}\label{\detokenize{generated/eflatun_uav.helpers:eflatun-uav-helpers}}\label{\detokenize{generated/eflatun_uav.helpers::doc}}\index{module@\spxentry{module}!eflatun\_uav.helpers@\spxentry{eflatun\_uav.helpers}}\index{eflatun\_uav.helpers@\spxentry{eflatun\_uav.helpers}!module@\spxentry{module}}\subsubsection*{Modules}


\begin{savenotes}\sphinxattablestart
\sphinxthistablewithglobalstyle
\sphinxthistablewithnovlinesstyle
\centering
\begin{tabulary}{\linewidth}[t]{\X{1}{2}\X{1}{2}}
\sphinxtoprule
\sphinxtableatstartofbodyhook
\sphinxAtStartPar
{\hyperref[\detokenize{generated/eflatun_uav.helpers.number_generators:module-eflatun_uav.helpers.number_generators}]{\sphinxcrossref{\sphinxcode{\sphinxupquote{eflatun\_uav.helpers.number\_generators}}}}}
&
\sphinxAtStartPar
This module creates numbers for given variable type of inputs
\\
\sphinxbottomrule
\end{tabulary}
\sphinxtableafterendhook\par
\sphinxattableend\end{savenotes}

\sphinxstepscope


\subsection{eflatun\_uav.helpers.number\_generators}
\label{\detokenize{generated/eflatun_uav.helpers.number_generators:module-eflatun_uav.helpers.number_generators}}\label{\detokenize{generated/eflatun_uav.helpers.number_generators:eflatun-uav-helpers-number-generators}}\label{\detokenize{generated/eflatun_uav.helpers.number_generators::doc}}\index{module@\spxentry{module}!eflatun\_uav.helpers.number\_generators@\spxentry{eflatun\_uav.helpers.number\_generators}}\index{eflatun\_uav.helpers.number\_generators@\spxentry{eflatun\_uav.helpers.number\_generators}!module@\spxentry{module}}
\sphinxAtStartPar
This module creates numbers for given variable type of inputs
\subsubsection*{Functions}


\begin{savenotes}\sphinxattablestart
\sphinxthistablewithglobalstyle
\sphinxthistablewithnovlinesstyle
\centering
\begin{tabulary}{\linewidth}[t]{\X{1}{2}\X{1}{2}}
\sphinxtoprule
\sphinxtableatstartofbodyhook
\sphinxAtStartPar
{\hyperref[\detokenize{generated/eflatun_uav.helpers.number_generators:eflatun_uav.helpers.number_generators.convert_string_to_float}]{\sphinxcrossref{\sphinxcode{\sphinxupquote{convert\_string\_to\_float}}}}}(string)
&
\sphinxAtStartPar
Converts a string to a deterministic random float representation between 0 and 1.
\\
\sphinxhline
\sphinxAtStartPar
{\hyperref[\detokenize{generated/eflatun_uav.helpers.number_generators:eflatun_uav.helpers.number_generators.convert_string_to_int}]{\sphinxcrossref{\sphinxcode{\sphinxupquote{convert\_string\_to\_int}}}}}(string, *{[}, base{]})
&
\sphinxAtStartPar
Converts a string to an deterministicly random integer representation using the specified base.
\\
\sphinxbottomrule
\end{tabulary}
\sphinxtableafterendhook\par
\sphinxattableend\end{savenotes}
\index{convert\_string\_to\_float() (in module eflatun\_uav.helpers.number\_generators)@\spxentry{convert\_string\_to\_float()}\spxextra{in module eflatun\_uav.helpers.number\_generators}}

\begin{fulllineitems}
\phantomsection\label{\detokenize{generated/eflatun_uav.helpers.number_generators:eflatun_uav.helpers.number_generators.convert_string_to_float}}
\pysigstartsignatures
\pysiglinewithargsret{\sphinxcode{\sphinxupquote{eflatun\_uav.helpers.number\_generators.}}\sphinxbfcode{\sphinxupquote{convert\_string\_to\_float}}}{\sphinxparam{\DUrole{n,n}{string}\DUrole{p,p}{:}\DUrole{w,w}{  }\DUrole{n,n}{str}}}{{ $\rightarrow$ float}}
\pysigstopsignatures
\sphinxAtStartPar
Converts a string to a deterministic random float representation between 0 and 1.

\sphinxAtStartPar
Works better for texts longer than 5 letters.
\begin{quote}\begin{description}
\sphinxlineitem{Parameters}
\sphinxAtStartPar
\sphinxstyleliteralstrong{\sphinxupquote{string}} (\sphinxstyleliteralemphasis{\sphinxupquote{str}}) \textendash{} The input string to be converted to a float.

\sphinxlineitem{Returns}
\sphinxAtStartPar
The float representation of the input string between 0 and 1.

\sphinxlineitem{Return type}
\sphinxAtStartPar
float

\end{description}\end{quote}
\subsubsection*{Example}

\begin{sphinxVerbatim}[commandchars=\\\{\}]
\PYG{g+gp}{\PYGZgt{}\PYGZgt{}\PYGZgt{} }\PYG{n}{convert\PYGZus{}string\PYGZus{}to\PYGZus{}float}\PYG{p}{(}\PYG{l+s+s2}{\PYGZdq{}}\PYG{l+s+s2}{Hello, World}\PYG{l+s+s2}{\PYGZdq{}}\PYG{p}{)}
\PYG{g+go}{0.3350260018341942}
\PYG{g+gp}{\PYGZgt{}\PYGZgt{}\PYGZgt{} }\PYG{n}{convert\PYGZus{}string\PYGZus{}to\PYGZus{}float}\PYG{p}{(}\PYG{l+s+s2}{\PYGZdq{}}\PYG{l+s+s2}{Hi, World?}\PYG{l+s+s2}{\PYGZdq{}}\PYG{p}{)}
\PYG{g+go}{0.8893743173684925}
\PYG{g+gp}{\PYGZgt{}\PYGZgt{}\PYGZgt{} }\PYG{n}{convert\PYGZus{}string\PYGZus{}to\PYGZus{}float}\PYG{p}{(}\PYG{l+s+s2}{\PYGZdq{}}\PYG{l+s+s2}{Hi, World}\PYG{l+s+s2}{\PYGZdq{}}\PYG{p}{)}
\PYG{g+go}{0.03764671504177386}
\end{sphinxVerbatim}

\end{fulllineitems}

\index{convert\_string\_to\_int() (in module eflatun\_uav.helpers.number\_generators)@\spxentry{convert\_string\_to\_int()}\spxextra{in module eflatun\_uav.helpers.number\_generators}}

\begin{fulllineitems}
\phantomsection\label{\detokenize{generated/eflatun_uav.helpers.number_generators:eflatun_uav.helpers.number_generators.convert_string_to_int}}
\pysigstartsignatures
\pysiglinewithargsret{\sphinxcode{\sphinxupquote{eflatun\_uav.helpers.number\_generators.}}\sphinxbfcode{\sphinxupquote{convert\_string\_to\_int}}}{\sphinxparam{\DUrole{n,n}{string}\DUrole{p,p}{:}\DUrole{w,w}{  }\DUrole{n,n}{str}}, \sphinxparam{\DUrole{o,o}{*}}, \sphinxparam{\DUrole{n,n}{base}\DUrole{p,p}{:}\DUrole{w,w}{  }\DUrole{n,n}{int\DUrole{w,w}{  }\DUrole{p,p}{|}\DUrole{w,w}{  }None}\DUrole{w,w}{  }\DUrole{o,o}{=}\DUrole{w,w}{  }\DUrole{default_value}{256}}}{{ $\rightarrow$ int}}
\pysigstopsignatures
\sphinxAtStartPar
Converts a string to an deterministicly random integer representation using the specified base.

\sphinxAtStartPar
Works better for texts longer than 5 letters.
\begin{quote}\begin{description}
\sphinxlineitem{Parameters}\begin{itemize}
\item {} 
\sphinxAtStartPar
\sphinxstyleliteralstrong{\sphinxupquote{string}} (\sphinxstyleliteralemphasis{\sphinxupquote{str}}) \textendash{} The input string to be converted to an integer.

\item {} 
\sphinxAtStartPar
\sphinxstyleliteralstrong{\sphinxupquote{base}} (\sphinxstyleliteralemphasis{\sphinxupquote{Optional}}\sphinxstyleliteralemphasis{\sphinxupquote{{[}}}\sphinxstyleliteralemphasis{\sphinxupquote{int}}\sphinxstyleliteralemphasis{\sphinxupquote{{]}}}\sphinxstyleliteralemphasis{\sphinxupquote{, }}\sphinxstyleliteralemphasis{\sphinxupquote{optional}}) \textendash{} The base to be used for the conversion. Defaults to 256.

\end{itemize}

\sphinxlineitem{Raises}
\sphinxAtStartPar
\sphinxstyleliteralstrong{\sphinxupquote{ValueError}} \textendash{} If the base is not an integer or if it is 0, \sphinxhyphen{}1, or 1.

\sphinxlineitem{Returns}
\sphinxAtStartPar
The integer representation of the input string.

\sphinxlineitem{Return type}
\sphinxAtStartPar
int

\end{description}\end{quote}
\subsubsection*{Example}

\begin{sphinxVerbatim}[commandchars=\\\{\}]
\PYG{g+gp}{\PYGZgt{}\PYGZgt{}\PYGZgt{} }\PYG{n}{convert\PYGZus{}string\PYGZus{}to\PYGZus{}int}\PYG{p}{(}\PYG{l+s+s2}{\PYGZdq{}}\PYG{l+s+s2}{Hello, World!}\PYG{l+s+s2}{\PYGZdq{}}\PYG{p}{)}
\PYG{g+go}{157}
\PYG{g+gp}{\PYGZgt{}\PYGZgt{}\PYGZgt{} }\PYG{n}{convert\PYGZus{}string\PYGZus{}to\PYGZus{}int}\PYG{p}{(}\PYG{l+s+s2}{\PYGZdq{}}\PYG{l+s+s2}{Hello, World}\PYG{l+s+s2}{\PYGZdq{}}\PYG{p}{)}
\PYG{g+go}{84}
\PYG{g+gp}{\PYGZgt{}\PYGZgt{}\PYGZgt{} }\PYG{n}{convert\PYGZus{}string\PYGZus{}to\PYGZus{}int}\PYG{p}{(}\PYG{l+s+s2}{\PYGZdq{}}\PYG{l+s+s2}{Hello, World!}\PYG{l+s+s2}{\PYGZdq{}}\PYG{p}{,} \PYG{n}{base} \PYG{o}{=} \PYG{l+m+mi}{36}\PYG{p}{)}
\PYG{g+go}{13}
\end{sphinxVerbatim}

\end{fulllineitems}



\chapter{Indices and tables}
\label{\detokenize{index:indices-and-tables}}\begin{itemize}
\item {} 
\sphinxAtStartPar
\DUrole{xref,std,std-ref}{genindex}

\item {} 
\sphinxAtStartPar
\DUrole{xref,std,std-ref}{modindex}

\item {} 
\sphinxAtStartPar
\DUrole{xref,std,std-ref}{search}

\end{itemize}


\renewcommand{\indexname}{Python Module Index}
\begin{sphinxtheindex}
\let\bigletter\sphinxstyleindexlettergroup
\bigletter{e}
\item\relax\sphinxstyleindexentry{eflatun\_uav}\sphinxstyleindexpageref{generated/eflatun_uav:\detokenize{module-eflatun_uav}}
\item\relax\sphinxstyleindexentry{eflatun\_uav.filters}\sphinxstyleindexpageref{generated/eflatun_uav.filters:\detokenize{module-eflatun_uav.filters}}
\item\relax\sphinxstyleindexentry{eflatun\_uav.helpers}\sphinxstyleindexpageref{generated/eflatun_uav.helpers:\detokenize{module-eflatun_uav.helpers}}
\item\relax\sphinxstyleindexentry{eflatun\_uav.helpers.number\_generators}\sphinxstyleindexpageref{generated/eflatun_uav.helpers.number_generators:\detokenize{module-eflatun_uav.helpers.number_generators}}
\end{sphinxtheindex}

\renewcommand{\indexname}{Index}
\printindex
\end{document}